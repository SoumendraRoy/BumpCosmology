% Define document class
\documentclass[modern]{aastex631}
\usepackage{showyourwork}

\newcommand{\todo}[1]{\textcolor{cyan}{TODO: #1}}

% Abbreviations
\newcommand{\mBH}{\ensuremath{m_\mathrm{BH}}}
\newcommand{\mBHMax}{\ensuremath{m_{\mathrm{BH},\mathrm{max}}}}
\newcommand{\mCO}{\ensuremath{m_{\mathrm{CO}}}}
\newcommand{\mCOMax}{\ensuremath{m_{\mathrm{CO},\mathrm{max}}}}
\newcommand{\mPISN}{\ensuremath{m_\mathrm{PISN}}}

% Begin!
\begin{document}

% Title
\title{Calibrated Cosmography With a Physical Model of the Black Hole Mass Function}

% Author list
\author[0000-0003-1540-8562]{Will M. Farr}
\affiliation{Department of Physics and Astronomy, Stony Brook University, Stony Brook NY 11794, USA}
\affiliation{Center for Computational Astrophysics, Flatiron Institute, New York NY 10010, USA}
\email{will.farr@stonybrook.edu}
\email{wfarr@flatironinstitute.org}

% Abstract with filler text
\begin{abstract}
    Lorem ipsum dolor sit amet, consectetuer adipiscing elit. Ut purus elit,
    vestibulum ut, placerat ac, adipiscing vitae, felis. Curabitur dictum
    gravida mauris, consectetuer id, vulputate a, magna. Donec vehicula augue eu
    neque, morbi tristique senectus et netus et. Mauris ut leo, cras viverra
    metus rhoncus sem, nulla et lectus vestibulum. Phasellus eu tellus sit amet
    tortor gravida placerat. Integer sapien est, iaculis in, pretium quis,
    viverra ac, nunc. Praesent eget sem vel leo ultrices bibendum. Aenean
    faucibus, morbi dolor nulla, malesuada eu, pulvinar at, mollis ac. Curabitur
    auctor semper nulla donec varius orci eget risus. Duis nibh mi, congue eu,
    accumsan eleifend, sagittis quis, diam. Duis eget orci sit amet orci
    dignissim rutrum.
\end{abstract}

% Main body with filler text
\section{Introduction}
\label{sec:intro}

% \begin{figure}
%     \begin{center}
%         \includegraphics[width=\columnwidth]{figures/m1-vs-m2.pdf}
%     \end{center}
%     \caption{Approximate population-weighted source frame masses from our set of detections.  For each event we show the 50\% and 90\% credible regions.  The source frame masses are computed assuming a \todo{cosmology}.}
%     \label{fig:m1-vs-m2}
%     \script{m1-vs-m2.py}
% \end{figure}

% \begin{figure}
%     \begin{center}
%         \includegraphics[width=\columnwidth]{figures/dNdm_PISN_effects.pdf}
%     \end{center}
%     \caption{Effect of changing various parameters in our model of the mass function.}
%     \label{fig:dNdm_PISN_effects}
%     \script{dNdm_PISN_effects.py}
% \end{figure}

% \begin{figure}
%     \begin{center}
%         \includegraphics[width=\columnwidth]{figures/dNdm_fitted.pdf}
%     \end{center}
%     \caption{Merger rate per primary mass at equal mass ratio and redshift zero inferred by our model fitted to the observations in O3.}
%     \label{fig:dNdm_fitted}
%     \script{dNdm_fitted.py}
% \end{figure}

% \begin{figure}
%     \begin{center}
%         \includegraphics[width=\columnwidth]{figures/cosmo_params_corner.pdf}
%     \end{center}
%     \caption{Corner plot of cosmological parameters and mass scale variables.}
%     \label{fig:cosmo-params-corner}
%     \script{cosmo_params_corner.py}
% \end{figure}

% \begin{figure}
%     \begin{center}
%         \includegraphics[width=\columnwidth]{figures/h_zoomin.pdf}
%     \end{center}
%     \caption{Posterior versus prior on $h$.}
%     \label{fig:h-zoomin}
%     \script{h_zoomin.py}
% \end{figure}

% \begin{figure}
%     \begin{center}
%         \includegraphics[width=\columnwidth]{figures/omh2_zoomin.pdf}
%     \end{center}
%     \caption{Posterior versus prior on $\omega_M = \Omega_M h^2$.}
%     \label{fig:omh2-zoomin}
%     \script{omh2_zoomin.py}
% \end{figure}

% \begin{figure}
%     \begin{center}
%         \includegraphics[width=\columnwidth]{figures/shape_corner.pdf}
%     \end{center}
%     \caption{Corner plot of the bump shape parameters fitted to O3.}
%     \label{fig:shape-corner}
%     \script{shape_corner.py}
% \end{figure}

% \begin{figure}
%     \begin{center}
%         \includegraphics[width=\columnwidth]{figures/peak-zoomin.pdf}
%     \end{center}
%     \caption{Shape of the 30 MSun peak inferred from O3.}
%     \label{fig:peak-zoomin}
%     \script{peak_zoomin.py}
% \end{figure}

% \begin{figure}
%     \begin{center}
%         \includegraphics[width=\columnwidth]{figures/mock_observation_corner.pdf}
%     \end{center}
%     \caption{An example of our mock posteriors for a random detected event.}
%     \label{fig:mock-observation-corner}
%     \script{mock_observation_corner.py}
% \end{figure}

% \begin{figure}
%     \begin{center}
%         \includegraphics[width=\columnwidth]{figures/peak-zoomin.pdf}
%     \end{center}
%     \caption{Shape of the 30 MSun peak inferred from O3.}
%     \label{fig:peak-zoomin}
%     \script{peak_zoomin.py}
% \end{figure}

% \begin{figure}
%     \begin{center}
%         \includegraphics[width=\columnwidth]{figures/mock_observation_corner.pdf}
%     \end{center}
%     \caption{An example of our mock posteriors for a random detected event.}
%     \label{fig:mock-observation-corner}
%     \script{mock_observation_corner.py}
% \end{figure}

\appendix

\section{Analytic Approximation}

Our model for the stellar-origin mass function rests on the map from core / ZAMS
mass to \emph{mean} BH mass, which we parameterize by 
\begin{equation}
    \mBH = \begin{cases}
        \mCO & 0 < \mBH < \mPISN \\
        \mBHMax - a \left( \mCO - \mCOMax \right)^2 & \mPISN \leq \mBH \leq \mBHMax
    \end{cases}
\end{equation}
where 
\begin{equation}
    a = \frac{1}{4\left( \mBHMax - \mPISN \right)}
\end{equation}
and
\begin{equation}
    \mCOMax = 2 \mBHMax - \mPISN
\end{equation}
are determined by the requirement that the map is smooth (continuous and with
continuous derivative) at $\mBH = \mPISN$.

\bibliography{bib}

\end{document}
